\documentclass[12pt]{article}
\usepackage[utf8]{inputenc}
\usepackage{natbib}
\usepackage{graphicx}
\usepackage{caption}
\usepackage{subcaption}
\usepackage{float}
\usepackage{amsmath}

\newcommand\abs[1]{\left|#1\right|}


% Purpose: Recap of first ex. TDT 4200
%
% Made by:
%	Even Florenes M.Sc Electronics NTNU 2016
%
% Last changes:
% 	2016-10-28 EF: Writing header and beginning to write document
%
% All rigths reserved Even Flørenæs NTNU 2016
%

\begin{document}

\thispagestyle{empty}
\mbox{}\\[6pc]
\begin{center}
\Huge{Problem 2 part 1}\\[2pc]

\Large{Even Flørenæs}\\[1pc]
\Large{Fall 2016}\\[2pc]

TDT4200 Parallel Computing\\
Department of Computer Science
\end{center}
\vfill

\newpage
\tableofcontents
\newpage
\section{Theory}
\subsection{Threads and processes}
A process is an executing instance of a program which can contain multiple threads. Threads are generally smaller processes in complexity than processes. A thread has shared memory while a process has a private memory blocks.
\subsection{Race condition}
When threads or processors attempt to simultaneously  access a resource, and the accesses can result in an error, the threads have a race condition. A race condition will leave the computation output depend on which thread "wins the race".
\subsection{Critical section}
A block of code that can only be executed by one thread at a time, to avoid a race condition, is called a critical section.
\end{document}